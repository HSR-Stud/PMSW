\section{Dokumentation \& Doxygen}
\subsection{Dokumenation}
\subsubsection{Zweck der Dokumentation}
\begin{itemize}
	\item Wissenssicherung / Wissenstransfer
	\item Kommunikation
	\item Sichtbarkeit des Projektfortschritts
\end{itemize}
\subsubsection{Dokumenation von Software}
\begin{itemize}
	\item Schnittstellenbeschreibung (API)
	\item Zielgruppe ist Programmierer
	\item Dokumente werden oftmals nicht nachgeführt, wenn Änderungen am Sourcecode gemacht werden $\rightarrow$ Inkonsistenz
	\item Im Sourcecode wird mittels speziellem gekennzeichnetm Kommentar Beschreibung erstellt /** Text */
	\item Mittels Dokumentationswerkzeugen kann Kommentar Source-Code extrahiert werden und daraus eine Beschreibung erstellen.  
\end{itemize}
\subsection{Doxygen}
Doxygen ist ein solches Dokumenationswerkzeugen wie oben erwähnt.\\
Es ist weit verbreitet und Open-Source. (Beispiel im Anhang)
\begin{itemize}
	\item unterstütze Sprachen: C,C++,Java, C\#,...
	\item Ausgabeformate: HTML; LaTex, RTF, XML,...
	\item auch UML-Diagramme möglich
	\item Command Line Tool
	\item existiert aber auch ein GUI und ein Eclipse-Plugin (Eclox, in Eclipse das Symbol @)
	\item benötigt Konfigurationsdatei Doxyfile 
	\item im File Doxyfile sind alle Einstellungen gespeicher was alles in die Dokumentation muss
\end{itemize}

\subsubsection{Doxygen Commands}
\renewcommand{\arraystretch}{1.5}
\begin{tabular}{|l|l|}
	\hline	/** DoxyCmt */ &  Doxygen Comment\\ 
	\hline	@file		& File Name. Next Line Description of the File.\\
	\hline	@author 	& Author\\
	\hline	@version	& Version \\
	\hline   @date		& Datum   \\ 
	\hline   @bug		& A known Bug \\
	\hline   @brief		& One Line description  \\
	\hline   @extended	& Description over several Lines  \\
	\hline   @param		& Description of your Parameter  \\
	\hline   @return		& Description of your Returnvalue     \\ 
	\hline   @warning	& Warnings \\ 
	\hline   @note		& Note\\ 
	\hline        
\end{tabular}

