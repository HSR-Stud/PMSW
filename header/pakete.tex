%--------------------%
%File für alle Pakete

\usepackage{array} % Extending the array and tabular environment
\usepackage{textcomp} % Wird für Copyright-Symbol,Währungen, Musikalische-Symbole benötigt
\usepackage{graphicx}
\usepackage{tabularx}
\usepackage{mathtools} % Das mathtools package ist eine Erweiterung zum amsmath package.
\usepackage{adjustbox} %adjustbox, minipage..
\usepackage[automark]{scrpage2} % Header und Footer
\usepackage{multirow} % Create tabular cells spanning multiple rows
\usepackage{multicol} % In­ter­mix sin­gle and mul­ti­ple columns
\usepackage{rotating} % Rotation tools, including rotated fullpage floats
\usepackage{xcolor}   
%Deutsche Sprache mit Sonderzeichen und Umlauten
\usepackage[utf8]{inputenc}  % UTF-8 unterstützung
\usepackage[T1]{fontenc} %ä,ü...
\usepackage[ngerman]{babel}  %Silbentrennung und Rechtschreibung Deutsch
\usepackage{longtable}

% Für Code einfügen
\usepackage{listings}

%Schriftart mit LuaLatex (alle Schriften aus Word möglich)
\usepackage{fontspec}
\setmainfont{Arial}

%Schriftart mit pdflatex Compiler
%\usepackage{helvet}
%\renewcommand{\familydefault}{\sfdefault}
%\fontfamily{phv}\selectfont

%Hyperlinks im Dokument
\usepackage[breaklinks,pdftex]{hyperref}

% Seitenränder für Formelsammlungen
\usepackage[left=1.27cm,right=1.27cm,top=0.80cm,bottom=0.80cm,includeheadfoot]{geometry}


%%%%%%%%%%%%%%%
% Code Layout %
%https://en.wikibooks.org/wiki/LaTeX/Source_Code_Listings
%%%%%%%%%%%%%%%

\definecolor{mygreen}{rgb}{0,0.6,0}
\definecolor{mygray}{rgb}{0.5,0.5,0.5}
\definecolor{mymauve}{rgb}{0.58,0,0.82}

\lstset{ %
    backgroundcolor=\color{white},   % choose the background color; you must add        \usepackage{color} or \usepackage{xcolor}
    basicstyle=\footnotesize,        % the size of the fonts that are used for the code
    breakatwhitespace=false,         % sets if automatic breaks should only happen at whitespace
    breaklines=true,                 % sets automatic line breaking
    captionpos=b,                    % sets the caption-position to bottom
    commentstyle=\color{mygreen},    % comment style
    deletekeywords={...},            % if you want to delete keywords from the given language
    otherkeywords={...},             % if you want to add more keywords to the set
    escapeinside={\%*}{*)},          % if you want to add LaTeX within your code
    extendedchars=true,              % lets you use non-ASCII characters; for 8-bits encodings only, does not work with UTF-8
    frame=single,	                 % adds a frame around the code
    keepspaces=true,                 % keeps spaces in text, useful for keeping indentation of code (possibly needs columns=flexible)
    keywordstyle=\color{blue},       % keyword style
    language=C++,                    % the language of the code   
    numbers=left,                    % where to put the line-numbers; possible values are (none, left, right)
    numbersep=5pt,                   % how far the line-numbers are from the code
    numberstyle=\tiny\color{mygray}, % the style that is used for the line-numbers
    rulecolor=\color{black},         % if not set, the frame-color may be changed on line-breaks within not-black text (e.g. comments (green here))
    showspaces=false,                % show spaces everywhere adding particular underscores; it overrides 'showstringspaces'
    showstringspaces=false,          % underline spaces within strings only
    showtabs=false,                  % show tabs within strings adding particular underscores
    stepnumber=2,                    % the step between two line-numbers. If it's 1, each line will be numbered
    stringstyle=\color{mymauve},     % string literal style
    tabsize=2,	                     % sets default tabsize to 2 spaces
   %title=\lstname                   % show the filename of files included with         \lstinputlisting; also try caption instead of title
}
\lstdefinestyle{customc++doxy}{
    belowcaptionskip=1\baselineskip,
    breaklines=true,
   %frame=L,
    xleftmargin=\parindent,
    language=C++,
    showstringspaces=false,
    basicstyle=\footnotesize\ttfamily,
    keywordstyle=\bfseries\color{blue},
    commentstyle=\itshape\color{mygreen},
    identifierstyle=\color{black},
    stringstyle=\color{gray},
    %TODO Change DOXYGEN Keyword-Highliting
     otherkeywords={           % DoxygenKeywords
         ...,
         ....,
         @mainpage,
         @file,
         @author,
         @version,
         @date,
         @bug,
         @brief,
         @extended,
         @param,
         @return,
         @warning,
         @note,
         @see},
}

%choose customstyle in DOC with \lstinputlisting[style=custom]{path}
%\lstset{escapechar=@,style=customc++}
\lstset{style=customc++doxy}
